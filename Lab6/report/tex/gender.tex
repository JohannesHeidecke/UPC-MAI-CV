\section{Gender recognition}

For this exercise we used a gender recognition system based on the AR-Face database. The used code and files are listed in appendix \ref{sec:ann}.

\subsection{Questions}

\question

\begin{itemize}
	\item Which is the information contained in ARFace.person?
	\item Why the size of the field internal, size(ARFace.internal), is $ 1188 \times 2210 $?
\end{itemize}

ARFace is a matlab struct. \texttt{ARFace.person} contains 2210 numeric values that link each image of a person in the database to a unique numeric identifier that corresponds to the person in the image. There are 85 different persons in the database, each occurring 26 times. 

The images are internally stored as bitmaps with a resolution of $33 \times 36$ pixels (which can be recalled from the field \texttt{ARFace.internalSz}. The images are reshaped and then saved as a vector of length $33 \times 36 = 1188 $. The size of the field \texttt{ARFace.internal} corresponds to 2210 different images of size 1188 pixel each. \newline


\question

\begin{itemize}
	\item What are the variables '\texttt{n}' and '\texttt{index}' of the function \texttt{fold\_validation.m}?
\end{itemize} 

The variables \texttt{n} and \texttt{index} are used to sample the data in different subsamples of training and test data. We are using F-fold validation, using F different, mutually exclusive subsamples as test data. The variable \texttt{n} is used in each iteration to pick $ \lfloor\frac{N}{F}\rfloor $ new persons to be used as test sample in this iteration, with $N$ being the number of unique persons and $F$ the number of iterations. The variable \texttt{index} links the matrix of images to a logical matrix with value $ 0 $ if the image is not of a person being used for the test sample in this iteration and $ 1 $ otherwise. The training and test data can then easily be set in each iteration based on this logical matrix. 

In the example of the 2110 images from the AR-Face database and a 10-fold validation, we use $ 10 $ subsamples as test data, each containing $ \lfloor\frac{N}{F}\rfloor = \lfloor\frac{85}{10}\rfloor = 8 $ images. The unique persons are shuffled and \texttt{n} contains $ 8 $ of them in each iteration in not overlapping windows. In the first iteration, \texttt{n} contains the $ 8 $ first persons of the shuffled list of unique persons, in the second iteration persons $9$ through $16$, and so on. The variable \texttt{index} maps the persons selected for the test data to the features and labels, splitting them in four different sets: \texttt{TrainSet, TrainLabels, TestSet, TestLabels}.\newline

\question

$ \bullet $ The function \texttt{main\_gender\_recognition.m} computes some evaluation measures obtained with `PCA'
($ dim= 5 $), `PCA95' (95\% variance explained) and `LDA'. Explain the meaning of these measures and discuss which is the best obtained result.